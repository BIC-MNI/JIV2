%
% $Header: /private-cvsroot/visualization/jiv/doc/inputs.tex,v 1.3 2000-04-13 17:20:38 crisco Exp $
%
% author: Chris Cocosco <crisco@bic.mni.mcgill.ca>
%

\section{Inputs}
\label{sec:inputs}


\subsection{Data files}
\label{sec:data-files}
Requirements for the 3D volume data read by JIV:
\begin{itemize}
\item the data file should be a sequence of $181 \times 217 \times 181
  = 7109137$ bytes, with no header, scanning the volume in $z, y, x$
  order (i.e.\ $z$ dimension changes fastest; this is also known as a
  ``transverse'' 3D data set); optionally, this file can compressed
  using \texttt{gnuzip} in which case its name needs to have the
  \texttt{.gz} suffix.
\item volumes are assumed to have the $(x,y,z)$ dimensions of
  $(181,217,181)$.
\item the $x, y, z$ axes and their positive direction are assumed to
  be (respectively): left to right, posterior to anterior, inferior to
  superior.
\item the voxel (``file'') to world coordinates transformation assumes
  that: voxels have an isotropic sampling of 1mm; the first voxel of
  the data file has a world coordinate $(x,y,z)$ of
  \mbox{$(-90,-126,-72)$mm}; voxel coordinates increase along the
  positive direction of their axis.
\end{itemize}

Note that the above volume sampling and associated world coordinates
correspond to the ``ICBM'' sampling and to the ``Talairach''
stereotaxic coordinates systems used at the MNI.  


\subsection{HTML code}
\label{sec:html-code}

JIV is a Java {\em applet}, thus needs through an HTML file; however,
this can be a very simple file, containing only the \verb+applet+ tag.
This applet needs to be given an run-time parameter (in the form of
the \verb+cfg+ or \verb+config+ applet parameter): the URL of a config
file, which contains information about which 3D data volumes to load,
the layout of the user interface, initial settings of various
controls, and so on.

The applet can be launched by some HTML code like this:
\begin{verbatim}
<applet height=50 width=400 archive="jiv.jar" code="jiv/Main.class">
        <param name="cfg" value="config_file">
</applet>
\end{verbatim}
where \verb+jiv.jar+ should be the URL of the JAR file containing the
JIV Java bytecode, and \verb+config_file+ should be the URL of the
appropriate JIV config file (\verb+config+ can be used instead of
\verb+cfg+ for the parameter name).


\subsection{Config file}
\label{sec:config-file}

The config file is expected to be in the following format:
\begin{itemize}
\item Lines that begin with \verb|#| or \verb|!| are comments and are
  ignored. 
\item Blank lines are ignored.
\item All other lines should specify a key/value pair and be of any of
  the following three equivalent forms:
\begin{verbatim}
         key = value
         key : value
         key value
\end{verbatim}
  Leading/trailing whitespace and control characters in \verb|value|
  are trimmed off.
\item The following escape characters are also recognized and
  treated as follows: 
  \begin{itemize}
  \item \verb|\newline| : 
    an escaped newline character is ignored, along with the spaces or
    tabs that follow it. 
  \item \verb|\n| : expands to a newline character.
  \item \verb|\r| : expands to a return character.
  \item \verb|\t| : expands to a tab character.
  \item \verb|\uxxxx| : expands to the Java Unicode character code
    specified by the hexadecimal digits.  
  \end{itemize}
\end{itemize} 

Keys which do not start with ``\verb+jiv.+'' represent {\em data
  volume aliases}, i.e.\ a kind of a short name for a 3D data volume
file. This is displayed as a title at the top of each panel, hence
it's a good idea to choose something descriptive and also short (such
that it will fit in the, possibly narrow, panel). The value associated
with the alias is data file's URL\@. Relative URL-s are interpreted
relative to the base URL of the HTML document from which the applet
was launched, and not relative to the base URL of the config file (not
exactly a feature \ldots).

Keys which start with ``\verb+jiv.+'' are JIV configuration
options. The following such keys are recognized:
\begin{itemize}
\item \verb+jiv.sync = [true|false]+ \\
  Sets the initial state of the \mbox{\em Sync all cursors}\ control.
  The default is false.
\item \verb+jiv.byte_values = [true|false]+ \\
  If true, all the voxel values, including the colormap range values,
  are presented in the user interface as byte values (0--255). If
  false, they are presented as fractional values (0.0--1.0). The
  default is false.
\item \verb+jiv.panel.N = alias+ \\
  Specifies an {\em individual volume panel}, i.e.\ an interface panel
  displaying a single data volume (specified by \verb+alias+, which
  should be a data volume alias declared somewhere else in the same
  config file).  \verb+N+ should be a non-negative integer and
  represents this panel's number.
\item \verb+jiv.panel.N.combine = alias1 alias2+ \\
  Specifies an {\em combined volume panel}, i.e.\ an interface panel
  displaying a combined view of two data volumes (specified by
  \verb+alias1+ and \verb+alias2+). \verb+N+ should be a non-negative
  integer and represents this panel's number. The two aliases should
  be separated by one or more blanks (\verb*+ +) or tabs (\verb+\t+).
  Also, these two aliases {\em should}\ be displayed in their
  individual panels as well. If an alias is displayed in more than one
  individual panel, then the lowest numbered such panel is used as the
  ``source'' for that volume alias.
\item \verb+jiv.panel.N.coding = [gray|grey|hotmetal|spectral|red|green|blue]+ \\
  Specifies the initial color coding for panel \verb+N+, which has to
  be an individual volume panel.
\item \verb+jiv.panel.N.range = L U+ \\
  Specifies the initial lower and upper limits of the color coding
  range for panel \verb+N+, which has to be an individual volume
  panel. \verb+L+ and \verb+U+ should be fractional (float) numbers in
  the range 0.0--1.0.  The two should be separated by one or more
  blanks (\verb*+ +) or tabs (\verb+\t+).
\end{itemize}
Panels are displayed left to right, sorted by their increasing number.
Note that these numbers do not have to be consecutive --- ``gaps'' in
the numbering sequence are silently skipped.  Note that the same alias
(i.e.\ the same data volume) can be displayed in several individual
and combined volume panels.

The config file is parsed in several passes, so the order of the
key/value pairs is irrelevant. However, if several conflicting
key/value definitions are given (which is bad practice, by the way),
not the last definition given but a random one of them will be
considered!


