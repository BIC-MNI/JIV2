%
% $Id: inputs.tex,v 1.9 2001-10-06 02:39:59 cc Exp $
%
% author: Chris Cocosco <crisco@bic.mni.mcgill.ca>
%

\section{Inputs}
\label{sec:inputs}


\subsection{Environment}
\label{sec:environment}

JIV can be run either as a Java {\em applet}, either as a (standalone)
Java application. 

Running it as an applet requires an HTML file to invoke it; however,
this can be a very simple file, containing only the \verb+applet+ tag.
This applet expects a run-time parameter (in the form of the
\verb+cfg+ or \verb+config+ applet parameter): the URL of a config
file, which contains information about which 3D data volumes to load,
the layout of the user interface, initial settings of various
controls, and so on.

The applet can be launched by some HTML code like this:
\begin{verbatim}
<applet height=50 width=400 archive="jiv.jar" code="jiv/Main.class">
        <param name="cfg" value="config_file">
</applet>
\end{verbatim}
where \verb+jiv.jar+ should be the URL of the JAR file containing the
JIV Java bytecode, and \verb+config_file+ should be the URL of the
appropriate JIV config file (\verb+config+ can be used instead of
\verb+cfg+ for the parameter name).

The HTML file is not needed when JIV is not run as an applet but as a
regular Java application. In this case, the config file's URL can be
supplied as a command line argument, or as the value of the \verb+cfg+
or \verb+config+ Java system properties (``environment variables'').
The top-level class that needs to be run is the same
(\verb+jiv/Main.class+).


\subsection{Config File}
\label{sec:config-file}

The config file is expected to be in the following format:
\begin{itemize}
\item Lines that begin with \verb|#| or \verb|!| are comments and are
  ignored. 
\item Blank lines are ignored.
\item All other lines should specify a key/value pair and be of any of
  the following three equivalent forms:
\begin{verbatim}
         key = value
         key : value
         key value
\end{verbatim}
  Leading/trailing whitespace and control characters in \verb|value|
  are trimmed off.
\item The following escape characters are also recognized and
  treated as follows: 
  \begin{itemize}
  \item \verb|\newline| : 
    an escaped newline character is ignored, along with the spaces or
    tabs that follow it. 
  \item \verb|\n| : expands to a newline character.
  \item \verb|\r| : expands to a return character.
  \item \verb|\t| : expands to a tab character.
  \item \verb|\uxxxx| : expands to the Java Unicode character code
    specified by the hexadecimal digits.  
  \end{itemize}
\end{itemize} 

Keys which start with ``\verb+jiv.+'' are JIV configuration options.
Keys which do not start with ``\verb+jiv.+'' give the location of the
image data files (image volumes). JIV configuration options refer to
data files by means of {\em data volume aliases}, i.e.\ a kind of a
short name.  The alias is displayed as a title at the top of each
panel, hence it's a good idea to choose something descriptive and
short (such that it will fit in the, possibly narrow, panel).

For a given alias ``myalias'', the value associated with key
\verb+myalias+ is the volume image file's URL (URL-s for the
individual slices are derived from the volume URL, as described in
section~\ref{sec:data-files}). The value associated with key
\verb+myalias.header+ is the URL of the header file associated with
the image file(s). All relative URL-s are interpreted relative to the
base URL of the config file.  If all volumes have the same header, it
is acceptable to specify it only for one of the volumes; otherwise, a
header should be specified for all volumes (see
section~\ref{sec:header-files} for the header file format).

The following JIV configuration options are supported:
\begin{itemize}
\item \verb+jiv.sync = [true|false]+ \\
  Sets the initial state of the \mbox{\em Sync all cursors}\ control.
  The default is false.
\item \verb+jiv.world_coords = [true|false]+ \\
  If false, only voxel coordinates will be available in the user
  interface (however, doing this is \emph{not} recommended practice!).
  The default is true.
\item \verb+jiv.byte_values = [true|false]+ \\
  If true, all the voxel values, including the colormap range values,
  are presented in the user interface as byte values (0--255). If
  false, they are presented as fractional values (0.0--1.0). The
  default is false.
\item \verb+jiv.panel.N = alias+ \\
  Specifies an {\em individual volume panel}, i.e.\ an interface panel
  displaying a single data volume (specified by \verb+alias+, which
  should be a data volume alias declared somewhere else in the same
  config file).  \verb+N+ should be a non-negative integer and
  represents this panel's number.
\item \verb+jiv.panel.N.combine = alias1 alias2+ \\
  Specifies an {\em combined volume panel}, i.e.\ an interface panel
  displaying a combined view of two data volumes (specified by
  \verb+alias1+ and \verb+alias2+). \verb+N+ should be a non-negative
  integer and represents this panel's number. The two aliases should
  be separated by one or more blanks (\verb*+ +) or tabs (\verb+\t+).
  Also, these two aliases {\em should}\ be displayed in their
  individual panels as well. If an alias is displayed in more than one
  individual panel, then the lowest numbered such panel is used as the
  ``source'' for that volume alias.
\item \verb+jiv.panel.N.coding = [gray|grey|hotmetal|spectral|red|green|blue]+ \\
  Specifies the initial color coding for panel \verb+N+, which has to
  be an individual volume panel.
\item \verb+jiv.panel.N.range = L U+ \\
  Specifies the initial lower and upper limits of the color coding
  range for panel \verb+N+, which has to be an individual volume
  panel. \verb+L+ and \verb+U+ should be fractional (float) numbers in
  the range 0.0--1.0.  The two should be separated by one or more
  blanks (\verb*+ +) or tabs (\verb+\t+).
\item \verb+jiv.download = [upfront|on_demand|hybrid]+ \\
  Specifies the data (down-)loading method. 

  In mode \verb+upfront+, all of the data is loaded and stored in
  memory before the user can view and interact with any of it. This
  guarantees the best interactive response of the viewer, however the
  user has to wait for all the data to load before the JIV interface
  becomes available.  This mode is recommended when the data is
  available locally, but is impractical for accessing remote data over
  a slow network.

  In mode \verb+on_demand+ image data is loaded one slice at a time,
  and only if and when the user is viewing that particular slice
  number.  This mode is recommended for remote access over very slow
  networks.  It minimizes the data downloads and the amount of memory
  required by JIV, but the interactive performance is completely
  dependent on the network speed.

  Mode \verb+hybrid+ combines the other two: the complete image volume
  is downloaded in the background, and the slices at the cursor are
  downloaded with priority, as soon as they are needed. This mode is
  the recommended for typical remote data access situations; it
  provides the fast startup of \verb+on_demand+ and, after the
  background downloading completes, the optimal interactive
  performance of \verb+upfront+.

\end{itemize}

Panels are displayed left to right, sorted by their increasing number.
Note that these numbers do not have to be consecutive --- ``gaps'' in
the numbering sequence are silently skipped.  Note that the same alias
(i.e.\ the same data volume) can be displayed in several individual
and combined volume panels.

The config file is parsed in several passes, so the order of the
key/value pairs is irrelevant. However, if several conflicting
key/value definitions are given (which is bad practice, by the way),
not the last definition given but a random one of them will be
considered!


\subsection{Data Files}
\label{sec:data-files}
\label{sec:header-files}
JIV reads 3D image data from an image~file which should contain the
image intensity (gray level) data represented as unsigned bytes
(8-bit). The image data are interpreted using an associated header
file, which specifies the volume sampling, world coordinates (real
world mm), and real image values.
All the data files can be optionally compressed using \texttt{gnuzip},
in which case their names (URL-s) need to have the \texttt{.gz}
suffix.  

The \emph{header file} is a text file with the same syntax as the
config file (see section~\ref{sec:config-file}). The following
statements are supported (where the 3 values of the right-hand side
can be separated by whitespace or ``\verb+,+'') :
\begin{itemize}
\item \verb+size : x_size y_size z_size+ \\
  the sizes (voxel counts) of the 3D image along the x, y, and z axes
  respectively.
\item \verb+start : x_start y_start z_start+ \\
  the distance (in real world mm) from the origin to the first voxel
  along the x, y, and z axes respectively.
\item \verb+step : x_step y_step z_step+ \\
  the signed distance (in real world mm) between the centers of
  consecutive voxels along the x, y, and z axes respectively; a
  negative value indicates that the file scans the volume in negative
  direction along that axis.
\item \verb+order : (permutation of {x,y,z})+ \\
  the order of dimensions in the file, i.e.\ the order in which the
  volume file scans the (3D) data volume --- 
  e.g.\ ``\verb+order : z y x+'' means that the $x$ coordinate changes
  fastest (volumes using this particular dimension ordering are also
  known as ``transverse'').
\item \verb+imagerange : range_min range_max+ \\
  the linear mapping from the byte voxel values to the real image
  voxel values: 0 maps to range\_min, 255 maps to range\_max.
\end{itemize}
The $x, y, z$ axes and their positive direction are assumed to be
(respectively): left to right, posterior to anterior, and inferior to
superior of a 3D medical image. Note that the sizes, starts, and steps
are always expected in \mbox{$x, y, z$} order, regardless of the file
dimension order. If any of the header information is not specified,
the following defaults are used (they correspond to the ``ICBM''
sampling and to the ``Talairach'' stereotaxic coordinates systems used
at the Montreal Neurological Institute, McGill University):
\begin{verbatim}
  size   :  181 217 181
  start  :  -90 -126 -72
  step   :  1 1 1
  order  :  z y x
  imagerange  :  0.0 1.0
\end{verbatim}

There are two kinds of image files: 
\begin{description}
\item[volume file:] contains the complete (3D) image volume data; it
  is required by the \verb+upfront+ and the \verb+hybrid+ download
  modes.
\item[slice file(s):] contain a 2D slice of the 3D image volume; three
  sets of all the slices orthogonal to each of the three coordinate
  axes are required by the \verb+on_demand+ and the \verb+hybrid+
  download modes.
\end{description}
The volume file URL is specified in the config file (as shown above in
section~\ref{sec:config-file}); the slice files are expected at URL-s
of the form: \texttt{base/orientation/slice\_number.extension}, where
\texttt{orientation} is one of ``01'', ``02'', or ``12'' --- the name
indicates which are the in-slice dimensions (in file dimension order)
for that orientation. For example, for a ``transverse'' volume
(\mbox{order: z y x}) 01, 02, 12 correspond to ``sagittal'' (z-y),
``coronal'' (z-x), and ``transverse'' (y-x) slice orientations.

\texttt{base} and \texttt{extension} are obtained by breaking the
volume file URL at the last ``\verb+.+'' (other than the suffixes
\verb+.gz+ or \verb+.bz2+); for example, all of the following:
\begin{verbatim}
  /some/dir/somename.raw_byte.gz
  /some/dir/somename.raw_byte
  /some/dir/somename.gz
  /some/dir/somename
\end{verbatim}
result in \texttt{base} = \texttt{/some/dir/somename} .

For converting MNI-MINC data to the JIV input format, two utilities
(Perl scripts) are distributed along with JIV: \verb+minc2jiv.pl+, and
\verb+jiv.pl+.


