%
% $Header: /private-cvsroot/visualization/jiv/doc/gui.tex,v 1.2 2000-04-12 02:16:26 cc Exp $
%
% author: Chris Cocosco <crisco@bic.mni.mcgill.ca>
%

\section{Graphical User Interface}
\label{sec:gui}

The main JIV window is composed of one or more ``panels'' --- columns
of interface elements separated by vertical lines.  Each panel has a
title indicating its content and type. There are two kinds of panels:
\begin{description}
\item[combined data volume panel] Has a title of the form
  ``\verb+vol_name1 <-> vol_name2+'', and displays a combined view of
  two 3D data volumes.
\item[invididual data volume panel] Has a title of the form
  ``\verb+vol_name+'', and displays a single 3D data volume.
\end{description}
The two data volumes that are displayed together in a combined volume
panel are always displayed in their individual panels as well.

JIV allows its main window to be resized at will (using the techniques
specific to your platform and windowing environment). However, if the
window is too small (especially if it's too narrow for the number of
displayed panels) some interface components may overlap in a confusing
way, or even become completely obscured.

%%%%%%%%%%%%%%%%%%%%%%%%%%%%%%%%%%%%%%%%%%%%%%%
\subsection{Common panel features}
\label{sec:common-features}
Each panel is composed of four visible main elements, aligned in a
vertical column; from top to bottom, they are:
\begin{enumerate}
\item ``Transverse'' 2D slice viewport ($Z = const$).
\item ``Sagittal'' 2D slice viewport ($X = const$).
\item ``Coronal'' 2D slice viewport ($Y = const$).
\item Panel controls area.
\end{enumerate}
The $x, y, z$ axes and their positive direction are (respectively):
left to right, posterior to anterior, and inferior to superior of the
head.  Besides the (always visible) controls area at the bottom of the
panel, additional controls are available through a panel-specific
popup menu.

\subsubsection{Slice viewport}

Inside a 2D slice viewport, the following mouse actions are available:
\begin{description}
\item[primary-mouse-button] {\em Moves the cursor}\ to a new position
  in slice's plane. Can also drag the cursor around.
\item[secondary-mouse-button] Combined with a vertical mouse movement,
  moves the cursor along the axis orthogonal to slice's plane
  (i.e.\ {\em changes the displayed slice}). The cursor displacement
  is proportional to the relative vertical mouse movement.
\item[Shift or Ctrl + primary-mouse-button] Moves the field of view
  (i.e.\ does a {\em pan}) by following the mouse movement/drag.
\item[Shift or Ctrl + secondary-mouse-button] Combined with a vertical
  mouse movement, {\em changes the zoom/scaling factor}.  The change
  in the scaled image dimensions is proportional to the relative
  vertical mouse movement.
\end{description}
The meaning of ``primary'' and ``secondary'' mouse buttons is provided
by the Java implementation running this applet, and is
platform-dependent --- e.g.\ on a Unix/X-Windows platform, with the
common right-handed mouse configuration, primary is the left button,
and secondary is either of the middle or right mouse buttons.  On all
platforms, the secondary button can be emulated by pressing the
\verb+Meta+ or \verb+Alt+ key together with the primary button.

Also, the following keys can be used for moving the cursor along the
coordinate axis orthogonal to slice's plane by an amount of 1mm
(one voxel):
\begin{itemize}
\item \verb+Right+($\rightarrow$) or \verb+Up+($\uparrow$) : positive
  increment (next slice).
\item \verb+Left+($\leftarrow$) or \verb+Down+($\downarrow$) :
  negative increment (previous slice).
\end{itemize}

The image scaling (i.e.\ zoom-up/down) is done using {\em
  nearest-neighbour}\ interpolation: pixels are replicated (for
enlargements) or skipped (for reductions) as needed.

When the viewport dimensions change (as a consequence of the main
window being resized), JIV will adjust the field of view such that
it's not less than the previous field of view, while at the same time
using as much of possible of the new viewport area.

\subsubsection{Controls area}
The controls area of each panel has, at the top, a group of three text
fields that display the current $X,Y,Z$ coordinates of the cursor in
that panel. These fields also allow editing (text input) -- the usual
text editing commands of your platform should work.  If a value out of
range is given, the field will revert to its previous (valid) value.
The coordinates displayed (and read in) are voxel or world
coordinates, depending on the current panel setting.

Any horizontal sliders present in this area are implemented using
platform-specific scrollbars, thus their behaviour in response to
mouse (and maybe keyboard) actions should be similiar to the other
scrollbars on your computer platform (and/or windowing system).

\subsubsection{Popup menu}
The mouse and/or keyboard command that brings up the popup menu is the
usual one for triggering context-sensitive (popup) menus on your
particular computer platform --- e.g.\ on Unix and on Microsoft
Windows, it's usually the right (second or third) mouse button.

The following menu actions are available in all panels:
\begin{itemize}
\item \verb+Coordinates type+ [choice] : Changes the type of
  coordinates that are displayed, and read in, in the controls area at
  the bottom.
\item \verb+Sync all cursors+ [toggle] : If \verb+on+, the cursor
  positions in all panels will be kept the same. When changing this
  control from \verb+off+ to \verb+on+, all cursors will be set to the
  cursor position of the first panel (from the left).
\item \verb+Quit+ : Closes the JIV window and exists the application.
  However, depending on the particular web browser or appletviewer you
  are using, this is probably not enough; in order to completely
  dispose of this running copy of the JIV applet you may have to move
  out of the HTML document that launched it, or maybe even close that
  browser frame/window. To make things worse, some web browsers don't
  release the (possibly large amount of) memory formerly used by the
  applet unless you completely shutdown the browser!
\end{itemize}

%%%%%%%%%%%%%%%%%%%%%%%%%%%%%%%%%%%%%%%%%%%%%%%
\subsection{Individual volume panel features}
\label{sec:individual-panel-features}
The 8-bit intensity values in the data volume are displayed in the
viewports using a user-controlled colormap.  This is composed of a
certain color-coding scheme (in between adjustable lower and upper
limits), an ``under'' color (below the lower color-coding limit) and
an ``over'' color (above the upper color-coding limit).

\subsubsection{Controls area}
To the left of the three coordinate fields, there's a read-only text
field displaying the voxel (intensity) value at the cursor position.

Below the coordinate text fields, there are controls for the lower
color-coding limit (left text display/entry field and lower slider)
and for the upper color-coding limit (right text display/entry field
and upper slider).  The lower limit can never be higher than the upper
limit. 

The values displayed (or read in) by the voxel intensity and
lower/upper color-coding limit text fields can be a fractional ones
(in the range 0.0--1.0) or byte values (in the range 0--255),
depending on how this instance of JIV was configured.

The current colormap is displayed as a color bar at the very bottom of
the panel.

\subsubsection{Popup menu}
The following menu actions are available in the individual volume
panels only:
\begin{itemize}
\item \verb+Color coding+ [choice] : Changes the color-coding scheme.
\item \verb+"Under" color+ [choice] : Changes the ``under'' color
  (\verb+(default)+ means the same color as the one at color-coding's
  lower limit).
\item \verb+"Over" color+ [choice] : Changes the ``over'' color
  (\verb+(default)+ means the same color as the one at color-coding's
  upper limit).
\item \verb+Tie colormap sliders+ [toggle] : If \verb+on+, the two
  color-coding limits behave like being connected together by a solid
  rod --- adjusting one value implies changing the other one such that
  their difference remains the same.
\end{itemize}

%%%%%%%%%%%%%%%%%%%%%%%%%%%%%%%%%%%%%%%%%%%%%%%
\subsection{Combined volume panel features}
\label{sec:combined-panel-features}
The colouring of each of the two data volumes is the one from the
individual volume panel representing that data volume. If a data
volume (i.e.\ a volume alias) is displayed by more individual volume
panels, then the left-most such panel is used as the coloring source
for that data volume.

Currently, the only method available for ``combining'' the two data
volumes is {\em blended}\ : the color of each pixel of the combined
image is given by
\[ color\_in\_volume\_1 \times ( 1 - \beta ) + 
color\_in\_volume\_2 \times \beta \] where $\beta$ is the blend factor
--- a fractional value in the range 0.0--1.0.

\subsubsection{Controls area}
Below the coordinate text fields, there is a blend factor ($\beta$)
slider, surrounded by two text display/entry fields: the left one for
$1 - \beta$, and the right one for $\beta$.  In other words, these two
text fields contain the weighting factors for the two combined data
volumes.


